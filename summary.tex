\documentclass{anstrans}
%%%%%%%%%%%%%%%%%%%%%%%%%%%%%%%%%%%
\title{Dynamic Resource Exchange Performance in Cyclus}
\author{Matthew J.~Gidden, Paul P. H.~Wilson}

\institute{
University of Wisconsin, Madison WI
}

\email{gidden@wisc.edu}

%%%% packages and definitions (optional)
\usepackage{graphicx} % allows inclusion of graphics
\usepackage{booktabs} % nice rules (thick lines) for tables
\usepackage{microtype} % improves typography for PDF

\newcommand{\Cyclus}{\textsc{Cyclus}}

\begin{document}

%%%%%%%%%%%%%%%%%%%%%%%%%%%%%%%%%%%%%%%%%%%%%%%%%%%%%%%%%%%%%%%%%%%%%%%%%%%%%%%%
\section{Introduction}
Nuclear fuel cycle simulation (FCS) is a field which seeks to model the
facilities and material flows required to produce nuclear power. Simulations
normally model a time span decades, or even centuries. Furthermore, myriad
decisions exist within a given fuel cycle simulation, such as the deployment
timing of facility types and deciding how material transfers should be executed
at a given time step. Accordingly, tradeoffs exist between the features provided
by a simulator and the performance of the simulator.

Historically, the most popular methodology used to model nuclear fuel cycles has
been system dynamics \cite{moisseytsev_dymond_2001, durpel_daness_2003,
  yacout_vision_2006, busquim_e_silva_system_2008}. In general, system dyanmics
has been found to be difficult to use when modeling complex, physics-dependent
supply chains. Attempts at modeling different fuel cycles have required altering
the underlying simulation engine \cite{guerin_impact_2009}. It is typically
difficult to add \textit{in situ} decision making, i.e., allowing a simulator to
make deployment or routing decisions dynamically, given its current
state. DANESS is the only simulator that claims any such capability in the
current literature \cite{van_den_durpel_daness_2009}. Finally, modeling regional
and institutional effects, i.e., allowing modeled relationships bewteen
management entities to determine material flow and deployment decisions, is a
feature absent from most simulators. 

The \Cyclus{} FCS \cite{cyclus2014} was designed to more easily model a variety
of fuel cycles. It uses an agent-based modeling paradigm in order to encapsulate
the difficulty of designing new agent archetypes. Archetypes define
parameterized agent logic and behavior, and can therefore be reused within and
between simulations. The core agent-interaction model in \Cyclus{} is the
Dynamic Resource Exchange (DRE) \cite{gidden_agent-based_2013,
  gidden_agent-based_2014}. The DRE, recomputed at each time step, polls the
supply and demand of commodities in the simulation and then determines the
trades to be executed between agents. Coupling the DRE with the \Cyclus{}
Region-Institution-Facility hierarchy model as well as the
archetype-prototype-agent model, highly dynamic, easily adjustable fuel cycles
can be modeled, addressing many of the issues developers and users have found
with system-dynamics-based models. However, \Cyclus{} must both be featureful
and performant. This paper provides a first-look at how the DRE model scales
with problem size by generating and solving a large number of
exchanges. Different solvers are analyzed including a \Cyclus{}-aware greedy
heuristic in addition to COIN-OR's LP and MILP solvers \cite{coinclp, coincbc}.

%%%%%%%%%%%%%%%%%%%%%%%%%%%%%%%%%%%%%%%%%%%%%%%%%%%%%%%%%%%%%%%%%%%%%%%%%%%%%%%%
\section{Methodology}
Method

%%%%%%%%%%%%%%%%%%%%%%%%%%%%%%%%%%%%%%%%%%%%%%%%%%%%%%%%%%%%%%%%%%%%%%%%%%%%%%%%
\section{Results and Analysis}
Results

%%%%%%%%%%%%%%%%%%%%%%%%%%%%%%%%%%%%%%%%%%%%%%%%%%%%%%%%%%%%%%%%%%%%%%%%%%%%%%%%
\section{Conclusions}
Conclusions

%%%%%%%%%%%%%%%%%%%%%%%%%%%%%%%%%%%%%%%%%%%%%%%%%%%%%%%%%%%%%%%%%%%%%%%%%%%%%%%%
\section{Acknowledgments}
Acks

%%%%%%%%%%%%%%%%%%%%%%%%%%%%%%%%%%%%%%%%%%%%%%%%%%%%%%%%%%%%%%%%%%%%%%%%%%%%%%%%
\bibliographystyle{ans}
\bibliography{bibliography}
\end{document}

